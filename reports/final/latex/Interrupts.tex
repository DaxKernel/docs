\chapter{Interrupts}\label{chapter:Interrupts}
Interrupts provide a quick and easy way for different parts of the OS to signal that important events 
that need attention have occurred. The basic idea here is that when something important that needs to 
be immediately handled occurs, the CPU stops what it is currently doing and calls another function.
This function is called an Interrupt Service Routine.

\vspace{1cm}
There are a total of 256 interrupts in the x86 hardware architecture. Of this, the first 32 are CPU exceptions.
The interrupts from 32-47 deal with various parts of the system such as keyboard and mouse. 
The remaining interrupts are user-defined.

\vspace{1cm}
When an interrupt occurs the CPU looks up the corresponding ISR from a table called Interrupt Descriptor Table.
It then branches to that function. ISRs are different from other functions in that they use the \textbf{iret} instruction
to return instead of the commonly used \textbf{ret} instruction. The difference is that while the latter only restores the IP,
the former also restores the flags and CS registers.
\pagebreak

\section{Programming Interrupts}\label{section:Programming Interrupts}

\subsection{Creating the IDT}\label{subsection:Creating the IDT}
An overview of the implementation is as follows:
\begin{enumerate}
    \item Each value in the IDT is represented by a C struct.
    \item The IVT itself is simply an array of C structs.
\end{enumerate}

Thus the C code for the IDT is as follows:
\begin{lstlisting}
    struct IDT_entry
    {
        uint16_t offset_lowerbits;
        uint16_t selector;
        unsigned char zero;
        unsigned char type_attr;
        uint16_t offset_higherbits;
    } IDT[IDT_SIZE];
\end{lstlisting}

\vspace{0.5cm}
The struct fields are described as follows:
\begin{itemize}
    \item \textbf{offset\_higherbits, offset\_lowerbits}:\\
    These together form the memory address of the ISR.
    \item \textbf{selector}:\\
    This is a code segment selector defined in the Global Descriptor Table (GDT).
    \item \textbf{zero}:\\
    This has to be zero by convention.
    \item \textbf{type\_attr}:\\
    This decides the type of interrupt gate (interrupt gate or task gate).
\end{itemize}
\pagebreak


\subsection{Creating the ISR}\label{subsection:Creating the ISR}

Now that the IDT has been created interrupt handlers can be created.\\
There are two options to proceed:
\begin{itemize}
    \item Use x86 assembly
    \item Use C with GCC extensions
\end{itemize}

The second choice is preferred because it gives the opportunity to write cleaner code.\\
However it must be noted that using C by itself is not an option because functions in C always end with the ret instruction.
GCC compiler provides additions to the C programming language that can sometimes be helpful in scenarios such as this.\\

Some C code to make the implementation easier is written:

\begin{lstlisting}
    struct interrupt_frame
    {
        uint32_t error_code;
        uint32_t ip;
        uint32_t cs;
        uint32_t flags;
    };
    #define IRQ_HANDLER __attribute__((interrupt)) void
    #define IRQ_ARG __attribute__((unused)) struct interrupt_frame *frame
\end{lstlisting}

The interrupt attribute is an GCC extension that will force the function to return with the \textbf{iret} instruction.

This setup makes it easy to create new ISRs in a very clean and readable fashion. For instance to create a ISR handler we can 
simply do:
\begin{lstlisting}
    IRQ_HANDLER gameControllerISR(IRQ_ARG)
    {
        // Logic here
    }
\end{lstlisting}
\pagebreak

\section{Loading interrupts}\label{section:Loading interrupts}
The final step is to tell the CPU where to find the Interrupt Descriptor Table.

\vspace{1cm}
This is done by the load\_idt() function:
\begin{lstlisting}
    static inline void load_idt()
    {
        typedef struct __attribute__((packed))
        {
            uint16_t limit;
            uintptr_t start;
        } IDT_REPRESENTATION;

        const IDT_REPRESENTATION rep = {(sizeof(struct IDT_entry) * (IDT_SIZE - 1)), (uintptr_t)IDT};

        asm("lidt %0; sti;"
            : /*No Output*/
            : "m"(*(IDT_REPRESENTATION *)&rep));
    }
\end{lstlisting}

\vspace{1cm}
The \textbf{lidt} instruction indicates the location of the IDT table to the CPU. This information if encoded as a C struct which
contains the starting address and the offset that must be added to reach the last IDT entry.
Extended asm is used to inline assembly within the C code. The packed attribute instructs GCC to not add any extra padding to the 
struct.\\
Finally \textbf{sti} instruction is used to enable interrupts.