\chapter{CPU Exceptions}\label{chapter:CPU Exceptions}
These are a kind of interrupt that signals that an exception or error has occurred. Most CPU exceptions are faults i.e they 
are recoverable. However some CPU exceptions are aborts i.e the OS cannot continue after they occur.

In DaxOS, every CPU exception leads to a kernel panic. Kernel panic involves showing a screen with the exception information
and then stopping the OS.

\section{Supported Exceptions}\label{section:Supported Exceptions}

DaxOS supports the following CPU exceptions:
\begin{itemize}
    \item \textbf{Divide By Zero}\\
    This occurs when a number is divide by zero.
    \item \textbf{Debug}\\
    This is triggered when single step mode is enabled.
    \item \textbf{Non Maskable Interrupt}\\
    This occurs in response to a non-recoverable hardware fault.
    \item \textbf{Breakpoint}\\
    This is triggered by running INT3 instruction.
    \item \textbf{Overflow}\\
    This occurs when the overflow bit in flag is set and INTO instruction is given.
    \item \textbf{Bound Range Exceeded}\\
    Occurs in response to a BOUND instruction.
    \item \textbf{Invalid Opcode}\\
    Triggered when a nonexistent opcode is used.
    \item \textbf{Device Not Available}\\
    Triggered when a Floating Point Unit is not available.
\end{itemize}

\section{Code Generation}\label{section:Code Generation}
Code for handling two cpu exceptions are given below:
\begin{lstlisting}
   IRQ_HANDLER cpu_exception_0(IRQ_ARG)
    {
        send_eoi();
        k_panic("Divide By Zero");
    }

    IRQ_HANDLER cpu_exception_1(IRQ_ARG)
    {
        send_eoi();
        k_panic("Debug");
    }
\end{lstlisting}
One obvious problem is that the code is duplicated. The two interrupt handlers are similar except for the string argument in 
k\_panic function.
This is a clear violation of the DRY principle which states that code must not be duplicated. 

\vspace{0.4cm}
DaxOS solves this issue by generating the duplicated code using a python script.
The python script produces the code needed for exception handling in two files: \\exception.c and exception.h, both 
of which can be configured to be produced at any location.