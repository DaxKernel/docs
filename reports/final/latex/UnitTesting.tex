\chapter{Unit Testing}\label{chapter:Unit Testing}

Testing is an integral practice to ensure and verify that the code written solves the problem that it is trying to solve.
Unit testing refers to testing the units of a program - in most cases these units are functions.

\section{Challenge}\label{section:Challenge}

An operating system is an example of a system software. It is difficult to write tests for such software.
Some related difficulties are:

\begin{itemize}
    \item Most unit testing frameworks depend on an existing standard C library.
    \item Lack of mature debugging utilities. 
    \item Difficulty in getting test results across to developers.
\end{itemize}

\section{Introducing DUnit}\label{section:Introducing DUnit}

To make testing easier, DaxOS uses a custom unit testing framework called \textbf{DUnit}.
The implementation is minimal and based on two assertions.
\pagebreak

The API can summarized as follows:

\vspace{0.5cm}
\begin{lstlisting}
    // Assert that a condition is true
    void D_UNIT_ASSERT(bool condition, const char *t_name);

    // Assert that a condition is false
    void D_UNIT_ASSERT_FALSE(bool condition, const char *t_name);
\end{lstlisting}
\vspace{1cm}

An example of an unit test written using DUnit is given below:
\vspace{0.5cm}
\begin{lstlisting}
    // Ensure that malloc returns NULL if we're out of memory
    void D_TEST_unfulfillable_request()
    {
        char *p = malloc(/* Quick way to get SIZE_MAX*/ (size_t)-1);
        D_UNIT_ASSERT_FALSE(p, "MEM/UNFULFILLABLE_MEM_REQUEST");
    }
\end{lstlisting}