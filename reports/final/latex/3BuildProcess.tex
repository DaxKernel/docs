\chapter{Build Process}\label{chapter:Build Process}

\section{Overview}\label{section:Overview}

DaxOS requires a certain number of tools for the build process:
\begin{itemize}
    \item \textbf{Cross-compiler}:\\
        This is a special kind of compiler that can output binaries for a different kind of computer architecture
        than the one in which it resides.

    \item \textbf{Binutils}:\\
        This contains other tools such as the assembler and linker.\\
        It also contains an archiver used to produce libraries.

    \item \textbf{Emulator}:\\
        Since an OS cannot be run as easily as other types of software, an emulator is used to accelerate the code-build-debug cycle.

    \item \textbf{Linux}:\\
        Although this is not a strict requirement, most of the tools specified above are easily available on Linux. \\
        Other OS such as Window can still be used although it is more time-consuming and non-intuitive.
\end{itemize}

\section{Building cross-compiler and binutils}\label{section:Building cross-compiler and binutil}

These tools cannot be downloaded as as binary package. Instead they must be compiled from source.
The buld process for these tools is somewhat complicated.\\
However in summary, the steps to be followed are:
\begin{enumerate}
    \item Download respective source files from the GNU website.

    \item Install the necessary dependencies using the package manager.
    For example on debian based distributions, the \textit{apt} package manager is used as follows:
    \begin{lstlisting}
    sudo apt install build-essentials, bison, flex, libgmp3-dev ...
    \end{lstlisting}

    \item Add path variables:
    \begin{lstlisting}
    export PREFIX="$HOME/opt/cross"
    export TARGET=i686-elf
    export PATH="$PREFIX/bin:$PATH"
    \end{lstlisting}

    \item Run configure scripts
    
    \item Issue make-install commands as follows:
    \begin{lstlisting}
        make all-gcc
        make all-target-libgcc
        make install-gcc
        make install-target-libgcc
    \end{lstlisting}

    \item Add newly compiled compiler binaries to \$PATH variable.

\end{enumerate}
\pagebreak

\section{Setting up the emulator}\label{section:Setting up the emulator}
DaxOS uses \textit{qemu} as its default and preferred emulator.\\
Qemu is fast, open-source and can be easily run from the terminal.
It can be easily installed using:
\begin{lstlisting}
    sudo apt install qemu-system-i386
\end{lstlisting}

\section{Overview of shell-scripts}\label{section:Overview of shell-script}
Building DaxOS is a complicated process that involves running many different commands
with a large number of optional parameters and flags. Therefore to make the build process easier,
DaxOS comes with a set of shellscript wrappers to automate the heavy build process.
With the shellscript, compiling the OS is as easy as invoking:
\begin{lstlisting}
    .\build.sh
\end{lstlisting}

Other available shellscripts and their functionalities were introduced in the Design Diagrams chapter.

\section{Makefile}\label{section:Makefile}
A Makefile is a particular type of file that specifies the steps to be taken to build a particular piece of software.\\
DaxOS uses Makefiles to drive the compilation process.\\
Makefiles are written by specifying a target and the steps needed
to produce that target. Additionally dependencies needed to build a target can also be specified, in which case, the steps needed
to produce the dependencies also need to be written.
\pagebreak

Consider the following snippet of a Makefile that is used in DaxOS:

\vspace{0.5cm}
\lstset{language=make}

\begin{lstlisting}
DaxOS.kernel: $(OBJS) $(ARCHDIR)/linker.ld
$(CC) -T $(ARCHDIR)/linker.ld -o $@ $(CFLAGS) $(LINK_LIST)
grub-file --is-x86-multiboot DaxOS.kernel

.c.o:
    $(CC) -MD -c $< -o $@ -std=gnu11 $(CFLAGS) $(CPPFLAGS)

.S.o:
    $(CC) -MD -c $< -o $@ $(CFLAGS) $(CPPFLAGS)
\end{lstlisting}

This makefile can be interpreted as follows:
\begin{itemize}
    \item To produce \textbf{DaxOS.kernel}, the files in \textit{OBJS} and \textit{linker.ld} in \textit{ARCHDIR} needs to be produced first.
    \item Object files(\textit{.o}) are produced from C source files (\textit{.c}) files by executing \$(CC) command.
    \item Object files(\textit{.o}) are produced from Assembly source files (\textit{.S}) files by executing \$(CC) command.
    \item \$(foo) indicates a variable named \textit{foo}.
\end{itemize}

\tcbox{
\textbf{Note}: \$(CC) here is a variable that is set to the GCC cross-compiler by the shellscript.
}